\documentclass[11pt, a4paper]{article}

\usepackage{amsmath, amssymb, titling}
\usepackage[margin=2.5cm]{geometry}
\usepackage[colorlinks=true, linkcolor=black, urlcolor=black, citecolor=black]{hyperref}
\usepackage{url}
\usepackage{graphicx}
\usepackage{caption}
\usepackage{subcaption}
\usepackage{float}
\usepackage{cancel}
\usepackage{fancyhdr, lastpage}
\usepackage{fourier-orns}
\usepackage{xcolor}
\usepackage{nomencl}
\makenomenclature
\usepackage{etoolbox}
\usepackage{sidecap}
\usepackage{adjustbox}
\usepackage{listings}
\usepackage{matlab-prettifier}
\usepackage[T1]{fontenc}

\sidecaptionvpos{figure}{c}
\setlength{\headheight}{18.2pt}
\setlength{\nomlabelwidth}{1.5cm}

\renewcommand\maketitlehooka{\null\mbox{}\vfill}
\renewcommand\maketitlehookd{\vfill\null}

\renewcommand{\headrule}{\vspace{-5pt}\hrulefill\raisebox{-2.1pt}{\quad\leafleft\decoone\leafright\quad}\hrulefill}
\newcommand{\parder}[2]{\frac{\partial {#1}}{\partial {#2}}}
% \renewcommand\nomgroup[1]{%
%   \item[\bfseries
%   \ifstrequal{#1}{F}{Far--Away Properties}{%
%   \ifstrequal{#1}{N}{Dimensionless Numbers}{%
%   \ifstrequal{#1}{M}{Matrices}{%
%   \ifstrequal{#1}{D}{Diagonals}{%
%   \ifstrequal{#1}{V}{Vectors}{%
%   \ifstrequal{#1}{P}{Dimensionless Average Properties}{}}}}}}
% ]}

\title{Intro to Turbulent Flow \\ HW1}
\author{Almog Dobrescu ID 214254252}

% \pagestyle{fancy}
\cfoot{Page \thepage\ of \pageref{LastPage}}

\begin{document}

\thispagestyle{empty}
\maketitle
\newpage

\pagenumbering{roman}
% \setcounter{page}{1}

\tableofcontents
\vfil
\listoffigures
\vfil
\lstlistoflistings
\newpage

\printnomenclature
\newpage

\pagestyle{fancy}
\pagenumbering{arabic}
\setcounter{page}{1}

\section{Computing a Turbulent Flow}
Considering a commercial airliner with a chord $L=5\left[\mathrm{m}\right]$ cruising at $U=250\left[\frac{\mathrm{m}}{\mathrm{sec}}\right]$.

\subsection{Estimating boundary layer thickness $\delta$}
According to Prandtl's one-seventh power law:
\begin{equation}
    \frac{\delta}{x}\approx\frac{0.16}{Re_x^\frac{1}{7}}
\end{equation}
At the trailing edge, the thickness of the boundary layer is:
\begin{equation}
    \begin{matrix}
        \displaystyle\frac{\delta}{L}\approx\frac{0.16}{Re_L^\frac{1}{7}} &,& \displaystyle Re_L=\frac{UL}{\nu}
    \end{matrix}
\end{equation}
To calculate the kinematic viscosity of air, let's assume the airliner is cruising at $30[\mathrm{kft}]$. At this hight, the temperature is about $-44.3^\circ[C]$ \cite{temp_distribution_at_atmosphere}. Hence, the kinematic viscosity is about $1\cdot10^{-5}$ \cite{air_properties}
\begin{equation}
    Re_L=\frac{250\cdot5}{1\cdot10^{-5}}=1.25\cdot10^8
\end{equation}
\begin{equation*}
    \Downarrow
\end{equation*}
\begin{equation}
    \delta\approx\frac{0.16\cdot5}{\left(1\cdot10^{-5}\right)^\frac{1}{7}}=0.0558[\mathrm{m}]
\end{equation}

\subsection{Smallest Dynamically Important Scales}
At the largest scales, the Reynolds number is very high, which indicates a turbulent regime. In a turbulent regime, there is no dissipation, which means:
\begin{equation}
 Re_\ell=\frac{u\ell}{\nu}\gg1
\end{equation}
However, we know that turbulent flows are dissipative, so there must be dissipation. For dissipation to accrue, the Reynolds number for the scales at which dissipation happens, the smallest scales, must be:
\begin{equation}
 Re_\eta=\frac{v\eta}{\nu}\sim1
\end{equation}
We can see that by fulfilling both demands for turbulent flow, there must be an energy transfer between scales, namely, an energy cascade.

\subsection{Number of Grid Points}
In order to accurately calculating the flow at the boundary layer, the size of one cell needs to be smaller then smallest eddy.
From the energy cascade concept, we can conclude that the rate of change of the kinetic energy at the large scales:
\begin{equation}
    \frac{du^2}{dt}\sim\frac{u^2}{\displaystyle\frac{\ell}{u}}\sim\varepsilon
    \label{eq: dissipation large scales}
\end{equation}
is balanced by the energy dissipated at the small scales. By dimensional analysis, we find that:
\begin{equation}
    \varepsilon\sim\nu\left(\frac{v}{\eta}\right)^2
    \label{eq: dissipation small scales}
\end{equation}
By combining Eq.\ref{eq: dissipation large scales} and Eq.\ref{eq: dissipation small scales} we get:
\begin{equation}
    \begin{matrix}
        \displaystyle\frac{\ell}{\eta}\sim{Re}_\ell^\frac{3}{4} & \text{and} & \displaystyle\frac{v}{u}\sim{Re}_\ell^{-\frac{1}{4}} & \text{and} & \displaystyle\frac{\displaystyle \frac{\ell}{u}}{\displaystyle \frac{\eta}{v}}\sim{Re}_\ell^\frac{1}{2}
        \label{eq: small scales properties}
    \end{matrix}
\end{equation}
From Eq.\ref{eq: small scales properties} we can calculate the size of one cell at the boundary layer:
\begin{equation}
    \eta_x\sim\frac{L}{{Re}_L^\frac{3}{4}}=\frac{5}{\left(1.25\cdot10^8\right)^\frac{3}{4}}=4.2295\cdot10^{-6}\left[\mathrm{m}\right]
\end{equation}
So, the number of grid point along the chord is:
\begin{equation}
    N_x\sim\frac{L}{\eta}=\frac{5}{4.2295\cdot10^{-6}}=1.1822\cdot10^{6}
\end{equation}
To calculate the number of grid point normal to the airfoil, we need to asstemate the small scale of the boundary layer:
\begin{equation}
    \eta_y\sim\frac{\delta}{{Re}_\delta\frac{3}{4}}=\frac{0.0558}{\left(\frac{U\delta}{\nu}\right)^\frac{3}{4}}=1.3745\cdot10^{-6}\left[\mathrm{m}\right]
\end{equation}
so the number of grid point normal to the airfoil:
\begin{equation}
    N_y\sim\frac{\delta}{\eta}=\frac{0.0558}{4.2295\cdot10^{-6}}=4.0574\cdot10^{4}
\end{equation}
The total number of grid point is therefore:
\begin{equation}
    N=N_xN_y=\boxed{4.7965\cdot10^{10}}
\end{equation}

\subsection{Instantaneous Flow Over One Eddy}
The turnover time of one eddy is defined as:
\begin{equation}
    t_{large}=\frac{\delta}{U}
\end{equation}
The time step needs to be smaller then the smallest time step. Therefore:
\begin{equation}
    t_{small}=\frac{\eta_y}{v}
\end{equation}
From Eq.\ref{eq: small scales properties} we get:
\begin{equation}
    \frac{t_{large}}{t_{small}}\sim{Re}_\delta^\frac{1}{2}
\end{equation}
\begin{equation*}
    \Downarrow
\end{equation*}
\begin{equation}
    N_t=\frac{t_{large}}{t_{small}}\sim{Re}_\delta^\frac{1}{2}\sim1223\text{ steps}
\end{equation}
\underline{I finished this quistion in about 3 hours}

\section{Heating a Room}
\subsection{Characteristic Time Scale}
The one dimensional heat equation for a non moving field is:
\begin{equation}
    \parder{T}{t}=\alpha\parder{^2T}{x^2}
\end{equation}
We will use dimensional analysis to determined the characteristic time scale for heating the room.
\begin{equation}
    \begin{array}{rcl}
        \displaystyle\frac{T_\infty}{t} & = & \displaystyle\alpha\frac{T_\infty}{L^2} \\\\
        t & = & \displaystyle\frac{L^2}{\alpha}
    \end{array}
\end{equation}
Assuming typical bedroom conditions \cite{air_thermal_diffusivity}:
\begin{equation*}
    \begin{matrix}
        \displaystyle\left.\alpha\right|_\text{$T=25^\circ[C]$}=22.39\cdot10^{-6}\left[\frac{\mathrm{m}^2}{\mathrm{sec}}\right] &,& L=3\left[\mathrm{m}\right]
    \end{matrix}
\end{equation*}
\begin{equation*}
    \Downarrow
\end{equation*}
\begin{equation}
    t=4.0197\cdot10^5\left[\mathrm{sec}\right]=4.6524\left[\mathrm{day}\right]
\end{equation}

\subsection{Induced Velocity Estimation}
The momentum equation under Boussinesq approximation can be written as:
\begin{equation}
    \frac{D}{Dt}\vec{u}=-\frac{1}{\rho}\nabla p+\nu\nabla^2\vec{u}+g\frac{\Delta T}{T}
\end{equation}
Where:
\begin{itemize}
    \item $u$ is velocity induced by a space heater via buoyancy effects
    \item $T$ is the local temperature
    \item $\Delta T$ is the difference between $T$ and the newly heated air
\end{itemize}
Assuming the buoyancy-driven flow is turbulent:
\begin{equation*}
    Re\gg1
\end{equation*}
The momentum equation can be rewritten as:
\begin{equation*}
    \parder{\vec{u}}{t}+\vec{u}\cdot\nabla\vec{u}=-\frac{1}{\rho}\nabla p+\nu\nabla^2\vec{u}+g\frac{\Delta T}{T}
\end{equation*}
and by substituting the operators we get:
\begin{equation}
    \parder{u}{t}+u\cdot\parder{u}{x}=-\frac{1}{\rho}\parder{p}{x}+\nu\parder{^2u}{^2x}+g\frac{T-T_h}{T}
\end{equation}
In order to estimate the magnitude of the induced velocity we will use dimensional analysis:
\begin{table}[H]
    \center
    \begin{tabular}{c|c|c|c|c}
        $u=u_\infty\tilde{u}$ & $t=T\tilde{t}$ & $x=h\tilde{x}$ & $p=\Lambda\tilde{p}$ & $T=T_\infty\tilde{T}$
    \end{tabular}
\end{table}
\begin{equation}
    \frac{u_\infty}{T}\parder{\tilde{u}}{\tilde{t}}+\frac{u_\infty^2}{h}\tilde{u}\cdot\parder{\tilde{u}}{\tilde{x}}=-\frac{\Lambda}{\rho h}\parder{\tilde{p}}{\tilde{x}}+\frac{\nu u_\infty}{h^2}\parder{^2\tilde{u}}{^2\tilde{x}}+g\frac{\tilde{T}-\displaystyle\frac{T_h}{T_\infty}}{\tilde{T}}
\end{equation}
Multiplying by $\displaystyle\frac{h^2}{\nu u_\infty}$:
\begin{equation}
    \frac{h^2}{\nu u_\infty}\frac{u_\infty}{T}\parder{\tilde{u}}{\tilde{t}}+\frac{h^2}{\nu u_\infty}\frac{u_\infty^2}{h}\tilde{u}\cdot\parder{\tilde{u}}{\tilde{x}}=-\frac{h^2}{\nu u_\infty}\frac{\Lambda}{\rho h}\parder{\tilde{p}}{\tilde{x}}+\frac{h^2}{\nu u_\infty}\frac{\nu u_\infty}{h^2}\parder{^2\tilde{u}}{^2\tilde{x}}+\frac{h^2}{\nu u_\infty}g\frac{\tilde{T}-\displaystyle\frac{T_h}{T_\infty}}{\tilde{T}}
\end{equation}
Multiplying by $\displaystyle\frac{h^2}{\nu u_\infty}$:
\begin{equation}
    \underbrace{\frac{u_\infty h}{\nu}}_\text{$Re_h$}\underbrace{\frac{h}{u_\infty T}}_\text{$St_h$}\parder{\tilde{u}}{\tilde{t}}+\underbrace{\frac{h u_\infty}{\nu}}_\text{$Re_h$}\tilde{u}\cdot\parder{\tilde{u}}{\tilde{x}}=-\frac{h\Lambda}{\rho\nu u_\infty}\parder{\tilde{p}}{\tilde{x}}+\parder{^2\tilde{u}}{^2\tilde{x}}+\frac{h^2}{\nu u_\infty}g\frac{\tilde{T}-\displaystyle\frac{T_h}{T_\infty}}{\tilde{T}}
\end{equation}

\section{Help Karl Pearson}

% Since the large scales appear to be in equilibrium from the point of view of the small scales:
% \begin{equation}
%     \begin{matrix}
%         \displaystyle\eta\sim\left(\frac{\nu^3}{\varepsilon}\right)^\frac{1}{4} & \text{and} & v\sim\left(\nu\varepsilon\right)^\frac{1}{4}
%     \end{matrix}
% \end{equation}

\bibliographystyle{ieeetr}
\bibliography{references}

\newpage
\appendix
\section{Listing of The Computer Program}
% \subsection{Parameters}
% \begin{lstinputlisting}[captionpos=b,stringstyle=\color{magenta},frame=single, numbers=left, style=MatLab-editor, basicstyle=\mlttfamily\small, caption={Parameters file},mlshowsectionrules=true]{./matlab/parameters.m}
% \end{lstinputlisting}

% \subsection{Main Code}
% \begin{lstinputlisting}[captionpos=b,stringstyle=\color{magenta},frame=single, numbers=left, style=MatLab-editor, basicstyle=\mlttfamily\small, caption={The main file},mlshowsectionrules=true]{./matlab/NM_hw1_CA.m}
% \end{lstinputlisting}

\end{document}
